\chapter*{Sommario}\label{ch:sommario}
Lo sviluppo recente di tecnologie come la guida autonoma e la robotica hanno evidenziato la crescente necessit� di una pi� accurata percezione dell'ambiente da parte dei dispositivi. La segmentazione semantica LiDAR ha recentemente attirato l'attenzione della ricerca industriale e accademica, grazie alla sua capacit� di comprendere in modo preciso il contenuto della scena percepita e agire direttamente sui dati grezzi forniti dai sensori. A tale scopo sono state proposte diverse soluzioni algoritmiche. RandLA-Net � stata scelta come architettura per questo lavoro, in quanto rappresenta una soluzione potente e leggera per gestire dati su larga scala. L'obiettivo della tesi � sviluppare una procedura per ottimizzare il processo di apprendimento  di RandLA-Net invece di andare a modificare la struttura della rete, mostrando come diverse tecniche di apprendimento possano migliorare le prestazioni a parit� di modello, dataset e risorse di calcolo. I metodi includono l'apprendimento curriculare (Curriculum Learning) e l'apprendimento contrastivo (Constrastive Learning), i quali mirano ad ottenere una migliore separazione tra le diverse classi, basandosi su un raggruppamento gerarchico a posteriori e fornendo una migliore comprensione del contenuto della scena. Strategie Coarse-to-Fine e metodi di regolarizzazione sono stati impiegati per tenere conto di un'organizzazione gerarchica delle classi. I risultati dei nostri metodi superano lo stato dell'arte, ottenendo un miglioramento di $1.5\%$ in termini di mIoU con diversi metodi. Hanno dimostrato la loro efficienza, fornendo un migliore equilibrio delle classi e una convergenza pi� rapida verso le prestazioni ottimali.