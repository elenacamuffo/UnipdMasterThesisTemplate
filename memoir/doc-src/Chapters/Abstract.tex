\chapter*{Abstract}\label{ch:abstract}
Recent advances in technologies such as autonomous driving and robotics have highlighted the growing need for precise environmental perception. LiDAR Semantic Segmentation has recently attracted the attention of industrial and academic research due to its ability to accomplish fine-grained scene understanding and act directly on raw content provided by sensors. The task has been approached in a variety of ways, so far. RandLA-Net architecture has been selected for this work as it represents a powerful and lightweight solution to deal with large-scale data. The goal of developing solutions that aim at optimizing the learning process rather than focusing on the architecture has been considered, showing how different learning techniques can be used to improve the performance of the model. The methods include Curriculum Learning and Contrastive Learning, which aim at having a better separation between different classes, providing a better understanding of the scene content. Coarse-to-Fine strategies have been developed as well as regularization methods in order to take into account a hierarchical organization of the classes. The results we obtained outperform the state of the art, achieving an improvement of $1.5\%$ in terms of mIoU with different methods. They proved their efficiency by providing a better balance of classes and a faster convergence to optimal performance.